\documentclass[12pt]{exam}
% general
\usepackage[utf8]{inputenc}
\usepackage[margin=1in]{geometry}
% math
\usepackage{amsmath, amssymb}
% tables
\usepackage{tabularx}
\usepackage{multicol}
% listings
\usepackage{color}
\usepackage[scaled=0.85]{sourcecodepro}
\usepackage{listings}

% exam parameters
\newcommand{\institution}{University of Victoria}
\newcommand{\course}{Fundaments of Programming with Engineering Applications}
\newcommand{\coursenumber}{CSC 111}
\newcommand{\courseid}{10691}
\newcommand{\sections}{B01}
\newcommand{\term}{Fall 2017}
\newcommand{\timelimit}{20 Minutes}
\newcommand{\examtitle}{Quiz 1}
\newcommand{\examdate}{November 15, 2017}
\newcommand{\examversion}{A}

% listings configuration
\definecolor{keywords}{RGB}{127,0,85}
\definecolor{comments}{RGB}{63,127,95}
\definecolor{strings}{RGB}{42,0,255}
\definecolor{frame}{RGB}{150,150,150}
\definecolor{numbers}{RGB}{100,100,100}
\lstdefinestyle{code}{
	language=C,
	tabsize=4,
	captionpos=b,
	showspaces=false,
	showtabs=false,
	breaklines=true,
	showstringspaces=false,
	breakatwhitespace=true,
	escapeinside={(*@}{@*)},
	commentstyle=\color{comments},
	keywordstyle=\bfseries\color{keywords},
	stringstyle=\color{strings},
	basicstyle=\small\ttfamily,
	frame=lines,
	rulecolor=\color{frame},
	xleftmargin=2em,
	framexleftmargin=1.5em,
	numbers=left,
	numbersep=10pt,
	numberstyle=\scriptsize\ttfamily\color{numbers}
}
\lstset{style=code}

% page configuration
\pagestyle{head}
\firstpageheader{}{}{}
\runningheader{\footnotesize \coursenumber}{\footnotesize \examtitle\ - Page \thepage\ of \numpages}{\footnotesize \examdate}
\runningheadrule

\begin{document}

% header
\noindent
\section*{\examtitle}
\textbf{\course} \\
{\footnotesize \coursenumber{} Section \sections. Time limit: \timelimit} \\
{\footnotesize \term{} (\examdate)} \\

% student information
\noindent
\begin{tabularx}{\textwidth}{|X|X|X|X|X|X|}
    \hline
    \small{Student name} & \small{} & \small{Student ID} & \small\bfseries{V00} & \small{Grade} & \small{} \\
    \hline
\end{tabularx}

\noindent \\
\rule[2ex]{\textwidth}{2pt}

\centering
{\footnotesize This exam is worth a total of \numpoints{} marks and contains \numquestions{} questions on \numpages{} pages.}

\begin{questions}

\question[2] What is the output of the syntactically correct C program below?
\vspace{0.3cm}
\begin{lstlisting}
#include <stdio.h>
#include <stdlib.h>
int main(void) {
	int k = 25;
	while (k < 50) {
		if (k%7 == 0) printf("%d ", k);
		k = k + 3;
	} /* while */
	printf("\n");
	return EXIT_SUCCESS;
} /* main */
\end{lstlisting}
\makeemptybox{1cm}
\addpoints

\question[20] Consider the function $f(x)=3x^3+2x^2+x+1$.
\noaddpoints % to omit double points count
\begin{parts}
\part[10]{} Calculate $f'(x)$.
\part[10]{} Calculate $f''(x)$.
\end{parts}
\addpoints

\question[2] One of these things is not like the others; one of these
things is not the same. Which one is different?
\begin{choices}
\choice John
\choice Paul
\choice George
\choice Ringo
\choice Socrates
\end{choices}

\question[2] One of these things is not like the others; one of these
things is not the same. Which one is different?
\begin{oneparchoices}
\choice John
\choice Paul
\choice George
\choice Ringo
\choice Socrates
\end{oneparchoices}

\question[3] Mark box if true.
\addpoints
\begin{checkboxes}
\choice 2+2=4
\choice $\frac{d}{dx} (x^2+1) = 2x+1$
\choice The Moon is made of cheese.
\end{checkboxes}

{%
\checkboxchar{$\Box$} % changing checkbox style locally
\question[3] Mark box if true.
\addpoints
\begin{checkboxes}
\choice 2+2=4
\choice $\frac{d}{dx} (x^2+1) = 2x+1$
\choice The Moon is made of cheese.
\end{checkboxes}
}%

{%
% changing choice items style locally
\renewcommand*\thechoice{\arabic{choice}} 
\renewcommand*\choicelabel{\thechoice)}
%
\question[2] Element with $Z=92$ is:
\begin{multicols}{2}
\begin{choices}
\choice H
\choice O
\choice F
\choice S
\choice Ba
\choice Pb
\choice U
\choice Pu
\end{choices}
\end{multicols}
}%

\question[10]
In no more than one paragraph, explain why the earth is round.
\makeemptybox{2in}

\question[20]
Explain blah, blah\ldots
\makeemptybox{\fill}

\newpage

\question[20]
Explain blah, blah\ldots
\fillwithlines{\fill}

\newpage

\question[20]
Explain blah, blah\ldots
\fillwithdottedlines{8em}

\end{questions}

\end{document}
